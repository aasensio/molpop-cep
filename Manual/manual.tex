\documentclass[12pt]{article}
\usepackage{natbib}
\usepackage[hypertex]{hyperref}
\usepackage{dirtree}


 \oddsidemargin 0.0in
 \evensidemargin 0.0in
 \textwidth 6.5in
 \topmargin -0.75in
 \textheight 9.5in
%  \pagestyle{empty}

%%%%%%%%%%%%%%%%%%%%%%%%%%%%%%%%%%%%%%%%%%%%%%%%%%%%%%%%%%%%%%%%%%%

\def\separation {0.5cm}
\def\non{\nonumber \\}
\def\DnuD     {\hbox{$\Delta\nu_D$}}
\def\Jbar     {\hbox{$\bar J$}}
\def\j        {\hbox{$\jmath$}}
\def\N        {\hbox{$\cal N$}}
\def\Ie       {\hbox{$I_e$}}
\def\Ji       {\hbox{$\bar J^i_\mathrm{ext}$}}
\def\about    {\hbox{$\sim$}}
\def\x        {\hbox{$\times$}}
\def\half     {\hbox{$1\over2$}}
\def\Ncr      {\hbox{$N'_{\rm cr}$}}
\def\mic      {\hbox{$\mu$m}}
\def\ion#1#2  {#1\,{\small {#2}} }
\def\tot      {\tau_t}
\def\t(#1){\tau^{#1}}
\def\a(#1){\alpha^{#1}}
\def\M{MOLPOP-CEP}
\def\LVG      {\texttt{LVG}}
\def\slab     {\texttt{slab}}

%%%%%%%%%%%%%%%%%%%%%%%%%%%%%%%%%%%%%%%%%%%%%%%%%%%%%%%%%%%%%%%%%%%


% For generation of the HTML manual with tth:
\def\tthdump#1{#1}      % For generating TeX source; ignored by tth
% Redefine symbols problematic for the browser:
%%tth:\def\ga{\hbox{$>\sim$}}
%%tth:\def\la{\hbox{$<\sim$}}
%%tth:\def\Mo{\hbox{$M_o$}}
%%tth:\def\Lo{\hbox{$L_o$}}
%%tth:\def\Mdot{\hbox{$M^{dot}$}}
%%tth:\def\Ivezic{Ivezic}

%%tth:\begin{html}<TITLE>User Manual for MOLPOP-CEP</TITLE>\end{html}
%%tth: This HTML file was generated from the TeX source by
%%tth: the translator TTH, which uses symbol fonts.  These fonts are
%%tth: not normally enabled for Netscape running under X, because of
%%tth: the way Netscape groups its fonts. If your browser has problems
%%tth: displaying the math symbols in this manual, an easy fix can be found
%%tth: on the TTH website at
%%tth:\begin{html}<A HREF="http://hutchinson.belmont.ma.us/tth/Xfonts.html">http://hutchinson.belmont.ma.us/tth/Xfonts.html</A>\end{html}
%%tth:\begin{html}<HR>\end{html}
%%%%%%%%%%%%%%%%%%%%%%%%%%%%%%%%%%%%%%%%%%%%%%%%%%%%%%%%%%%%%%%%%%%

\begin{document}

\title                  {\sc User Manual for \M}

\author{ A. Asensio Ramos \\
         Instituto de Astrof\'{\i}sica de Canarias\\
         38205, La Laguna, Tenerife, Spain\\ \\
        Moshe Elitzur \\
        Department of Physics and Astronomy \\
        University of Kentucky, Lexington, KY 40506-0055
        \\[0.5in] \today}% \today}
\date{}
\maketitle


\tableofcontents

\newpage

\section*{Disclaimer}

This software is distributed ``as is'' and the authors do not take any
responsibility for any consequence derived from its use. Use it with care and
never trust the output without a careful meditation. 

This code is copyrighted,
1976--2008 by Moshe Elitzur and Andr\'es Asensio Ramos, and may not be copied
without acknowledging its origin. Use of this code is not restricted, provided
that acknowledgement is made in each publication. The bibliographic reference
to this version of \M\ Elitzur \& Asensio Ramos 2006, MNRAS 365, 779. Send bug
reports, comments and questions to any of the authors.


\newpage

%%%%%%%%%%%%%%%%%%%%%%%%%%%%%%%%%%%%%%%%%%%%%%%%%%
%%%%%%%%%%%%%%%%%%%%%%%%%%%%%%%%%%%%%%%%%%%%%%%%%%
\section{Introduction}

\M\ is a code for the exact solution of radiative transfer problems in
multi-level atomic systems. The novel contribution of the code is that the
radiative transfer equations is analytically integrated so that the final
problem is reduced to the solution of a non-linear algebraic system of
equations in the level populations. The radiative transfer is solved
analytically using the Coupled Escape Probability formalism presented by
\cite{elitzur_asensio_cep06} and summarized in the last chapter of this manual.
The current version of the code is limited to plane-parallel slabs that can
present arbitrary spatial variations of the physical conditions.

The code is written in standard Fortran 90. It is based on the MOLPOP code
written by M. Elitzur that used single-zone escape probabilities for the
solution of the radiative transfer problem. The original MOLPOP code, written
in Fortran 77, has been ported to Fortran 90. During the translation, the code
has been modularized and all common blocks have been moved to external modules
that are used only where necessary. All the machinery present in MOLPOP for
reading the input file and carry out all the needed calculations (interpolation
of the collisional rates, selection of the active levels, etc.) are still
present. The fundamental idea when merging together the MOLPOP code and the CEP
code was to maintain the large flexibility of the input already present in
MOLPOP. When the solution method chosen in the input file is the single-zone
escape probability, the original MOLPOP code is executed. When CEP is chosen as
the method, the routines belonging to the CEP code are used. Although the
resulting code is a mixture of two existing codes, the interface between both
is simple and robust.

%%%%%%%%%%%%%%%%%%%%%%%%%%%%%%%%%%%%%%%%%%%%%%%%%%
%%%%%%%%%%%%%%%%%%%%%%%%%%%%%%%%%%%%%%%%%%%%%%%%%%
\section{Uncompressing and compiling MOLPOP-CEP}

The latest version of the code is placed on the Github repository at
\begin{verbatim}
https://github.com/aasensio/molpop-cep
\end{verbatim}
You can download it
in two different ways. The most convenient is using \texttt{git}, if installed in
your system. Just type:
\begin{verbatim}
git clone https://github.com/aasensio/molpop-cep
\end{verbatim}
and the code will be downloaded on the \texttt{molpop-cep} directory in the
current directory. An alternative way is by pressing the button ``Download ZIP''
on the repository webpage and uncompress it. If you clone it, you can always
pull new changes that we make to the code by typing
\begin{verbatim}
git pull.
\end{verbatim}
In any case, the MOLPOP-CEP directory
will contain the master input file \texttt{molpop.inp} (see below) and the
following subdirectories:

\begin{enumerate}
\item
{\tt Source} contains the Fortran 90 sources and a simple makefile.
\item
{\tt DataBase} contains molecular data files with energy levels,
A-coefficients and collision rates
\item
{\tt Samples} contains sample input files in separate directories for the
respective molecular species. Each of these directories includes a
sub-directory \texttt{OutputTest} with the output files these inputs produce.
{\tt Manual} contains this manual.
\end{enumerate}

The code has been tested on several Linux platforms using the Intel Fortran
Compiler (\texttt{ifort}) and the GNU Fortran Compiler (\texttt{gfortran}). The
source code is in the \texttt{Source/} directory. The compilation is performed
with the supplied \texttt{makefile}. It is quite simple and easy to modify, and
contains additional comments about compiling and pre-processing flags. The
default compiler is the free \texttt{gfortran} and you can use any other
compiler through the variable \texttt{COMPILER}. To compile the code, type:
\begin{verbatim}
       make clean
       make
\end{verbatim}
After compiling and linking, the executable is copied to the MOLPOP-CEP
directory that contains the master input \texttt{molpop.inp}. Running the
program should produce output in the subdirectories of {\tt Samples}. You can
check your output against those in the corresponding {\tt OutputTest}
subdirectory.

\begin{quote}
{\bf Note:} The MOLPOP-CEP executable can be placed anywhere as long as it is
run from a directory that contains \texttt{molpop.inp}.
\end{quote}


%%%%%%%%%%%%%%%%%%%%%%%%%%%%%%%%%%%%%%%%%%%%%%%%%%
%%%%%%%%%%%%%%%%%%%%%%%%%%%%%%%%%%%%%%%%%%%%%%%%%%
\section{General input}


%%%%%%%%%%%%%%%%%%%%%%%%%%%%%%%%%%%%%%%%%%%%%%%%%%%
\subsection{General considerations}
A single MOLPOP-CEP run can process an unlimited number of models, each of which
can correspond to a different molecular species. To accomplish this, MOLPOP-CEP's
input is always the master input file \texttt{molpop.inp}, which lists the names of the
actual input files for all models. These filenames must have the form
\texttt{fname.inp}, where \texttt{fname} is arbitrary and can include a full path, so that a
single run may produce output models in different directories (as is the case
with the distribution). In \texttt{molpop.inp}, each input filename must be listed on a
separate line, with the implied extension \texttt{.inp} omitted. Make sure you press the
``Enter" key after every filename you enter, especially if it is in the last
line of \texttt{molpop.inp}. Empty lines are ignored, as is all text following the ``\texttt{\%}''
sign. This enables you to enter comments and conveniently switch on
and off the running of any particular model.

The input files have a free format, text and empty lines can be entered
arbitrarily. All lines in these files that start with the ``\texttt{*}'' sign are echoed in
the output, and can be used to print out notes and comments. This option can
also be useful when the program fails for some mysterious reason and you want
to compare its output with an exact copy of the input line as it was read in
before processing by \M. The occurrence of relevant numerical input, which is
entered in standard Fortran conventions, is flagged by the equal sign
``\texttt{=}''. The only restrictions are that all required input entries must
be specified, and in the correct order; the most likely source of an input
error is failure to comply with these requirements.  Recall, also, that Fortran
requires a carriage return termination of the file's last line if it contains
relevant input. Single entries are always preceded by the equal sign,
``\texttt{=}'', and terminated by a blank, which can be optionally preceded
with a punctuation mark.  For example: \texttt{T = 10,000 K} as well as
\texttt{Temperature = 1.E4 degrees} and simply \texttt{t = 10000.00} are all
equivalent, legal input entries (note that comma separations of long numbers
are permitted).  Because of the special role of ``\texttt{=}'' as a flag for
input entry, care must be taken not to introduce any ``\texttt{=}'' except when
required.  All text following the ``\texttt{\%}'' sign is ignored and this can
be used to comment out material that includes ``\texttt{=}'' signs.  For
example, different options for the same physical property may require a
different number of input entries. By commenting out with ``\texttt{\%}'', all
options may be retained in the input file with only the relevant one switched
on.

\subsection{Directory structure}
The following is the directory structure.
\dirtree{%
.1 molpop/.
.2 Basecol/.
.2 DataBase/.
.3 Coll/.
.2 DataBaseBasecol/.
.3 Coll/.
.2 Manual/.
.2 Samples/.
.3 C+/.
.3 CO/.
.3 H2O/.
.3 OH/.
.3 SiO/.
.2 Source/.
}

We recommend the input files defined in \texttt{molpop.inp} to be located in the \texttt{Samples} directory, separated
in subdirectories for a better organization.

%%%%%%%%%%%%%%%%%%%%%%%%%%%%%%%%%%%%%%%%%%%%%%%%%%
\section{Input files}

The input contains three types of data --- physical parameters, numerical
accuracy parameters and output control.  You can use the supplied input files
as templates. In order to show the structure of an input file, we take
\texttt{Samples/CO/Flower\_rates.inp} as example and consider all the input
parameters one by one.

\subsection{Solution Method}

\M\ can solve the level population problem either in the standard escape
probability approximation or with the exact CEP method. The actual method is
chosen as the first input:

\vspace{\separation}
\begin{verbatim}
    Method (LVG, LVGPP, slab, CEP-NEWTON, CEP-NAG or CEP-ALI) = CEP-ALI
\end{verbatim}
The first three options use the escape probability approximation for large
velocity gradients (in spherical and plane-parallel geometries) or a quiescent slab. The other options invoke the full CEP
technique with three different solution methods. CEP-NEWTON uses a standard
Newton method for directly solving the nonlinear statistical equilibrium
equations \ref{eq:statis_equil_eq} of the CEP formalism. CEP-NAG is similar,
but uses a convex combination of Newton and scaled gradient directions to
ensure global convergence. In both cases, the derivatives of the Jacobian are
calculated analytically as a consequence of the analytical treatment of the CEP
method. CEP-ALI uses the accelerated $\Lambda$-iteration approach, iterating
from the calculation of the radiation field at each zone with the aid of eqs.\
(\ref{eq:pi}) and the solution of the preconditioned linearized statistical
equilibrium eqs.\ (\ref{eq:statis_equil_eq}).

In the case of LVG calculations, you need to provide the expansion velocity
instead of the local linewidth and one must specify an additional entry, the
logarithmic velocity gradient:
%
\smallskip
\begin{verbatim}
     dlogV/dlogr = 1.0 (in effect only for LVG)
\end{verbatim}


\subsection{Molecular Database}

Enter the path to the directory {\tt DataBase}, containing all the molecular
data, from where \M\ is run. For example, if {\tt DataBase} is a subdirectory
of the directory containing \texttt{molpop.inp} (which is the case for the
zipped package) then the input is

\vspace{\separation}
\begin{verbatim}
            Data directory = DataBase
\end{verbatim}
% (Mention the sample case with a different directory).

\subsection{The Molecule}

The list of available molecules can be found in the file
\texttt{DataBase/mol\_list.dat} (see \S\ref{sec:basecol}). Choose one name
from that list and enter the number of energy levels to be included in the run;
this number should not exceed the maximum listed in
\texttt{DataBase/mol\_list.dat}:

\vspace{\separation}
\begin{verbatim}
        Molecule data file = CO     
        Number of energy levels = 11
\end{verbatim}

Currently, the database contains one species, SiO, with data for the lowest 19
rotational levels in the first four vibration states. In this case \M\ corrects
for the finite number of rotational levels in the ground vibrational level, as
described in \cite{lockett_elitzur92}.  The input is then

\vspace{\separation}
\begin{verbatim}
        rot_correction = on    
        jmax = 19
\end{verbatim}
Otherwise, set 
\vspace{\separation}
\begin{verbatim}
        rot_correction = off
\end{verbatim}
if you do not want to apply the correction.

\subsection{Collision Rates}
\label{sec:collisions}

As noted above (\ref{sec:physical}), the collision rate between any two levels
is $nK_{ij}$, where $K_{ij}$ (cm$^3$\,s$^{-1}$) is the rate coefficient for the
transition and $n$ is the overall density. \M\ can handle an arbitrary number
of collisional parters. The contribution of each one to the overall collision
rate is $fnK_{ij}$, where $f$ is the fractional contribution of the particular
collider, i.e., $fn$ is the collider density (in cm$^{-3}$). These fractional
abundances are entered as weight factors with arbitrary scale, and \M\ takes
care that the normalizations add up to 1. It is also possible to scale each
collisional rate independently by a factor so that it is possible to experiment
what happens when one of the collisional partners is neglected or its
collisional efficiency is increased by an arbitrary factor. There are a number
of options for entering the individual rate coefficients $K_{ij}$ themselves.


\subsubsection{Tabulated Cross Sections}

Collision rates are input from tabulations in files. You need to specify
the name of the data file listing the collisional rate between each pair of
levels at a set of different temperatures; a file with this name must exist in
subdirectory {\tt Coll} of the database directory specified in the second input
entry (see \S\ref{sec:basecol} below). \M\ interpolates between these temperatures if the
input value falls between two tabulated ones. If the prescribed temperature is
outside the tabulated range, \M\ will extrapolate according to the selection
made with the next keyword. When \texttt{CONST} is selected, \M\ will use the
same value as the largest or smallest tabulated temperature, as appropriate.
Selecting \texttt{SQRT(T)} invokes an extrapolation proportional to $\sqrt{T}$
from the appropriate border. Example:

\vspace{\separation}
\begin{verbatim}
            Number of collision partners = 2
               weight / columns for CEP = 1
                               data file = CO_H2o.kij
        extrapolation (SQRT(T) or CONST) = sqrt(T)
                weight / columns for CEP = 1
                               data file = CO_H2p.kij
        extrapolation (SQRT(T) or CONST) = sqrt(T)
\end{verbatim}
This input will produce collisions with an equal mix of ortho- and para-H$_2$,
with collision rates from \cite{flower01}. The option \texttt{weight / columns for CEP}
indicates the weighting applied to each individual collider. In case of CEP, you can
select the column of the collider from the file.

\subsection{Physical Conditions}
\label{sec:physical}

When the CEP method is invoked, the code can handle a slab with variable
physical conditions. The parameters for each zone of the slab are read from a
specified input file. The file contains a header of arbitrary length limited by
a line containing just the ``\texttt{$>$}'' symbol. Subsequent lines list for
each zone the (1) width in cm; (2) overall density in cm$^{-3}$; (3)
temperature in K; (4) molecular abundance; (5) local linewidth in km\,s$^{-1}$.
The last quantity accounts for the local value of \DnuD\ in the Doppler profile
$e^{-[(\nu - \nu_0)/\Delta \nu_D]^2}$. If the local thermal velocity
$\sqrt{2kT/m}$ exceeds $c\DnuD/\nu_0$, the code will use it instead.

Example: the following input
%
\smallskip
\begin{verbatim}
     File listing physical conditions = Samples/CO/Flower_rates.physical
\end{verbatim}
invokes entry of physical conditions from the file
\texttt{Samples/CO/Flower\_rates.physical} listed here:
\begin{verbatim}
     Sample file with a slab with variable physical conditions

     >

     Number of zones = 5

  Width          n          T       Abundance     Linewidth      n(o-H2)/n  n(p-H2)/n
   [cm]       [cm^(-3)]     [K]                      [km/s]
   2.d15       1.d4       100.d0      1.d-4          1.d0         0.75       0.25
   2.d15       1.d4       110.d0      1.d-4          1.d0         0.75       0.25
   2.d15       1.d4       120.d0      1.d-4          1.d0         0.75       0.25
   2.d15       1.d4       110.d0      1.d-4          1.d0         0.75       0.25
   2.d15       1.d4       100.d0      1.d-4          1.d0         0.75       0.25
\end{verbatim}
Note that the density $n$ enters the calculation in two ways: (1) The collision
rate (s$^{-1}$) between levels $i$ and $j$ is $C_{ij} = nK_{ij}$, where
$K_{ij}$ (cm$^3$\,s$^{-1}$) is the rate coefficient for the transition (see \S
\ref{sec:collisions} below). (2) The molecular number density (cm$^{-3}$) is
the product of $n$, listed in the second column, and the abundance, which is
listed in the fourth. The columns from the sixth and above display the relative abundance
of the colliders that will be considered in the collisions. The number of columns
have to be equal to the number of colliders.


When the code is invoked in the escape probability approximation (solution
method is either LVG or slab), the physical conditions are uniform. In that
case it is possible to use
%
\smallskip
\begin{verbatim}
     File listing physical conditions = none
\end{verbatim}
and enter the physical conditions directly in the input file as follows:
\vspace{\separation}
\begin{verbatim}
     Gas Temperature = 100 K
     n = 1.0e4  cm^-3
     Molecular abundance (n_mol/n) = 1.0e-4
     Velocity = 1 km/sec (linewidth for slab)
\end{verbatim}
You can use this input method for physical conditions also in the case of CEP
calculations if the physical conditions in the slab are uniform.

\begin{quote}
{\bf Note:} If the physical conditions are entered from a file, make sure your
input {\em DOES NOT} contain the last four (or five) entries. You can leave
them in the input file and just comment them out with the symbol {\tt \%}.
\end{quote}

There are some special radiative transfer effects that are only implemented
in escape probability calculations:
\begin{verbatim}
    Dust properties file = DataBase/Standard_ISM.dat
    Dust abundance = 0.    relative to standard ISM abundance; 
                           when not 0, dust absorption effects are included 
                           The following two input parameters are in effect 
                           only when the dust abundance is not 0
\end{verbatim}

Then, you need to provide the molecular abundance and the linewidth:
\begin{verbatim}
     Molecular abundance (n_mol/n) = 1.0e-4
     Velocity = 1 km/sec (linewidth for slab)
\end{verbatim}
You can use this input method for physical conditions also in the case of CEP
calculations if the physical conditions in the slab are uniform.

\subsection{Line Overlap}
Photons generated in a certain transition can be sometimes absorbed in another
if the linewidths exceed the frequency separation between the lines. This
process is known as line overlap or line fluorescence, and plays an important
role in OH transitions and the Bowen fluorescence phenomenon. Currently, \M\
can handle this effect only when invoked in the escape probability mode (\LVG\
or \slab). The method of calculation is described in \cite{lockett_elitzur89}.
To invoke this calculation use \vspace{\separation}
\begin{verbatim}
        Line overlap (on/off) = on
\end{verbatim}
For example, the input file \texttt{Samples/OH/Offer\_rates.inp} carries out a
calculation taking into account the effect of line overlap. Otherwise, enter
{\tt overlap = off}.


\subsection{Maser Lines}
When the optical depth of a line becomes negative, the intensity at this
transition is amplified by the medium in a maser effect. The following input
determines whether the effect of the maser radiation on the level populations
(the ``saturation effect") is taken into account:

\vspace{\separation}
\begin{verbatim}
        Maser saturation (on/off) = no
\end{verbatim}
Selecting {\tt yes} invokes an escape probability calculation for the
saturation effect \citep[see \S 5.3 in][]{elitzur92}. This calculation can be
selected only if the solution method is \LVG\ or \slab. All CEP calculations
must have {\tt no} for the saturation entry. Calculations with the {\tt no}
option produce the maser pump and loss rates that can be used to estimate maser
emission from the standard phenomenological maser model (see \S\ref{sec:output} below).




% \subsubsection{Hard-Sphere Collisions}

% This collision law, invoked with the keyword \texttt{HARD\_SPHERE} is available
% for all molecules. In this analytic approximation, all downward collision rate
% per sub-level are equal to the the same value $\sigma v_{\rm T}$, where
% $\sigma$ is a common input cross section and $v_{\rm T}$ is the local thermal
% speed. The sample input file \texttt{OH/Hard\_Sphere.inp} is an example:

% \vspace{\separation}
% \begin{verbatim}
%             Number of collision partners = 1
%                                   weight = 1
%                   collision rates option = hard_sphere
%                                x-section = 1.e-15 cm^2
%                           scaling factor = 1.0
% \end{verbatim}

% \subsubsection{SiO Rotation-Vibrations}

% This collision law, invoked with the keyword \texttt{SiO\_ROVIB} is available
% only for SiO. It involves SiO excitations in collisions with H$_2$, using the
% approximate theory of \cite{bieniek_green83}. This option is utilized in the sample
% input file \texttt{SiO/SiO.inp}:

% \vspace{\separation}
% \begin{verbatim}
%             Number of collision partners = 1
%                                   weight = 1
%                   collision rates option = SiO_ROVIB
%                           scaling factor = 1.0
% \end{verbatim}



\subsection{External Radiation}

The cosmic microwave background (CMB) blackbody radiation at a temperature of
$T_\mathrm{CMB}=2.725$ K is always included. In addition, a number of other
radiation fields can be added and in each case the radiation can illuminate
either side of the source or both. Furthermore, the molecules can be immersed
in a radiation bath with the specified spectral shape. This is specified with
the keywords \texttt{left, right, both} and \texttt{internal}. Note that
distances into the slab are measured from its left side.

\subsubsection{Diluted Blackbody}

An arbitrary number of external diluted blackbody terms, $WB_\nu(T)$, can be
specified. Each term is parameterized by its temperature \texttt{T\_bb} and
dilution factor \texttt{W}. When several radiation fields are used, remember to
always end the list with the \texttt{W = 0}:

\vspace{\separation}
\begin{verbatim}
             W = 0.1
             T_bb = 2500 K
             Illumination from (left/right/both/internal) = left
             W = 0.
\end{verbatim}

\subsubsection{Single-Temperature Dust}

The radiation field of dust with the uniform temperature $T$ is $B_\nu(T)(1 -
e^{-\tau_\nu})$, where $\tau_\nu$ is the dust overall optical depth at
frequency $\nu$.  You specify such a radiation field by its temperature $T$ and
dust optical depth at visual. From the latter, \M\ determines the optical depth
at every wavelength from a tabulation in the file {\tt dust.dat}, which must be
kept in the parent directory, together with {\tt molpop.inp}. The dust
properties correspond to standard ISM dust. You can use a different dust by
substituting the provided tabulation with one of your own.

\M\ can handle an arbitrary number of such radiation fields. The list is
terminated by an entry with zero optical depth:

\vspace{\separation}
\begin{verbatim}
             Dust tau at visual = 3.
             Dust temperature = 100 K
             Illumination from (left/right/both/internal) = internal
             Dust tau at visual = 0.
\end{verbatim}

\subsubsection{Radiation Field from a File}

Any spectral shape for the radiation field can be entered as a tabulation in a
file, whose name is specified as input. The tabulation should have a header, with
its end marked with a line with \texttt{>} (see \texttt{Samples/ISRF.dat} for an example)
and then list the
normalized spectral shape $\lambda F_\lambda/F_{\rm bol}\ (= \nu F_\nu/F_{\rm
bol})$, where $F_{\rm bol} = \int F_\lambda d\lambda$ is the bolometric flux,
as a function of wavelength $\lambda$ in \mic. The flux level at the molecular
source is specified through the overall luminosity of the radiation source and
the distance to it:

\vspace{\separation}
\begin{verbatim}
  Radiation from file = none
%           Type of entry for the bolometric scale (ENERGY_DEN/L&R) = L&R
%           Jbol = 3.E-6 W/m^2
%           Luminosity = 1.E4 Lo  
%           Distance from source = 1.E15 cm
%           Illumination from (left/right/both/internal) = internal
\end{verbatim}
Two possible options are available for the bolometric scale. Either \texttt{ENERGY\_DEN}, so then
a bolometric radiation density should be provided with \texttt{Jbol} or \texttt{L\&R}, i.e., luminosity
and distance.
To signal the end of spectral input files or that there is no input from a file
altogether enter
\begin{verbatim}
             Radiation from file = none
\end{verbatim}

\subsection{Solution Controls}

Whichever solution method is invoked, \M\ solves the non-linear level
population equations with an iteration method---either Newton or accelerated
$\Lambda$-iterations. The solution accuracy and an upper bound on the number of
allowed iterations are controlled as follows:
%
\begin{verbatim}
       Accuracy in solution of the equations  = 1.0e-3
       Maximum number of iterations to solve  = 50
\end{verbatim}
You may try to solve the level populations for a desired input straightforward
without any attempt to improve the convergence. In that case, enter
\begin{verbatim}
       Solution strategy = fixed
\end{verbatim}
However, this may cause numerical difficulties. To aid convergence, \M\ offers
a step-by-step strategy. Starting from a situation in which the solution is
known, the code changes the column density in small increments using the
solution from the previous step as the initial guess for the next one. The
exact solution is readily obtained in two limiting cases. When all optical
depths are zero (the optically thin limit) the statistical rate equations for
the level populations are linear and can be easily inverted. The solutions
serve as the starting point for an {\tt increasing} strategy, in which the
column density increases up to a desired value. When all optical depths are
very large, all the level populations thermalize, following the Boltzmann
distribution at the local temperature. Thermal populations serve as the
starting point for the {\tt decreasing} strategy, in which the column density
decreases until the desired value is reached. In each case you have to specify
an initial target that serves as the smallest/largest value for all optical
depths of the initial solution. The next three entries control the sizes of the
steps. The initial step size, entered as the number of steps per decade,
controls also the intervals for output printing. If at any point the step size
is too large, \M\ will keep decreasing it until it either achieves a successful
solution\footnote{If a successful solution is achieved with a smaller step size
than originally prescribed, \M\ will attempt to gradually increase the step
size once it has passed the rough spot that caused problems. The number of
steps per decade will never be smaller than the initial one.} or bumps against
either of the two limits set in the input: a lower limit on the step size,
entered through the maximum number of steps allowed per decade, and an upper
limit on the total number of steps allowed throughout the entire run. The
variation of column density is terminated when either physical dimension or the
overall column density reaches a prescribed upper/lower limit. Here's an
example of the {\tt increasing} strategy:

\vspace{\separation}
\begin{verbatim}
       Solution strategy = increasing
       start with all optical depths less than tau_m        = 1
       Initial number of steps per decade                   = 4
       Maximum number of dimension steps allowed per decade = 20
       Total number of steps allowed to reach any limit     = 100
       stop when dimension exceeds R_m                      = 1.0E20 cm
       or total column exceeds N_col                        = 1.E26 cm^-2
\end{verbatim}
In a {\tt decreasing} input, {\tt tau\_m} would be a large optical depth
providing a lower limit for all transitions in the initial solution while
{\tt R\_m} and {\tt N\_col}, the stopping criteria for the convergence process,
would have values smaller than the desired target.

When solving the radiative transfer problem using the CEP method, it is
possible to let the code use a grid refinement technique to approach the exact
solution of the problem. This grid refinement systematically increases the
number of zones in which each original zone is subdivided until the maximum
relative change in the level populations between one grid and the next finer
grid falls below a given threshold. The code also allows the user not to apply
the grid refinement technique and solve the radiative transfer problem in the
given grid. This strategy is chosen by using a zero:

\vspace{\separation}
\begin{verbatim}
   Precision in CEP grid convergence (0 to ignore convergence) = 1.d-1
\end{verbatim}

\section{Which files call each other}
\begin{itemize}
\item \texttt{molpop.inp} defines the input file for the calculation, usually found in the directory \texttt{Samples}.
\item The keyword \texttt{Data directory} defines the root directory where the molecular data is stored.
\item The keyword \texttt{Molecule} inside each configuration file defines the file that contains the molecular information, which is found in the directory defined
by the previous keyword.
\item The keyword \texttt{File listing physical conditions} defines the physical properties of the slab when a CEP calculation with variable physical conditions is carried out.
\item The keyword \texttt{data file} in the definition of the collision partners refers to the specific table with collisional data in the \texttt{Coll} directory
inside the directory defined by the \texttt{Data directory} keyword.
\item The keyword \texttt{Radiation from file} defines the spectral shape of the boundary condition.
\end{itemize}




\section{Output}
\label{sec:output}
\M\ offers a printout of parts, or all, of the molecular data. This allows you
to check that the data were entered properly. Printing is controlled with {\tt
on} and {\tt off} switches:
\begin{verbatim}
      Print energy level data      = off
        statistical weights g_i    = on
        energy in cm^{-1}          = on
        energy in GHz              = on
        energy in K                = on
        quantum numbers            = on
      Print transition data        = on
        wavelength in micron       = on
        energy in GHz              = on
        energy in K                = off
        Einstein coefficients A_ij = on
        collision rates C_ij       = on
      Stop after printing molecular data = off
\end{verbatim}
In this example, the energy level data will {\em not} be printed, even though
individual switches are on, because the master switch for these data is off.
The last switch even allows you to just produce a data tabulation without
running \M\ at all. Also, when the physical conditions vary in the slab, a
printout of the collision rates, if requested, will cover only the first zone.

Next you control the amount of output the program produces during the run. The
different printing options are controlled with {\tt on} and {\tt off} switches;
in the following example, all options are turned off by the master switch:
\begin{verbatim}
      Progress print control                          = off
        Messages from NEWTON                          = on
        Messages from step-size selector              = on
        Messages from CEP progress                    = on
        Printing output for initial guess             = on
\end{verbatim}
Note that whenever the CEP method is selected, \M\ will report its progress on
the screen irrespective of the output switches; that is, the above listed
switch only controls the printed output.

The default output, always produced by \M, contains a summary as follows:
\begin{verbatim}
      R        Tot column  mol column   emission
      cm          cm-2        cm-2     erg/s/mol
   1.000E+16   1.000E+20   1.000E+16   8.557E-21
   1.778E+16   1.778E+20   1.778E+16   8.288E-21
   3.162E+16   3.162E+20   3.162E+16   7.928E-21
\end{verbatim}
The last column is the overall cooling rate from all the transitions included
in the calculation. Additional output can be produced by turning on appropriate
switches. For every printing step, you can output: (1) Detailed level
populations for all the transitions. (2) Information on all inverted lines, if
any exist. This output will include the optical depth, excitation temperature
and inversion efficiency. (3) Information on the prescribed number of top
thermal emitting lines (note, there are no more then $\half N\x N$ transitions
among $N$ levels!):
\begin{verbatim}
      Information on each step                            = on
        print detailed populations                        = off
        information on all inverted lines                 = on
        Number of cooling lines to print (0 to bypass)    = 15
\end{verbatim}
Finally, you can select some specific transitions for special output that will
be listed at the end of the run in summary form. Enter a number different from
0 (up to 10) for the number of transitions and then enter a pair of level
identifier numbers for each transition.  The following example selects a
summary output for the $J = 1 \to 0$ and $J = 7 \to 6$ rotational transitions
of CO:
\begin{verbatim}
      the number of transitions = 2
       i =  2      j = 1
       i =  8      j = 7
\end{verbatim}
For each transition, the output lists the excitation temperature, optical depth
and the line emission (erg/s/mol). If the transition is inverted, the last
quantity is listed as zero. Instead, the output lists quantities relevant for
the standard, phenomenological maser model \citep[see][]{elitzur92}: The
inversion efficiency, pump rate for each level, loss rate for each level and
their average. Note that all optical depths in the outputs are at line
center, so that if you want the total optical depth, you have to multiply
it by $\sqrt{\pi}$.


The input file \texttt{fname.inp} produces the output file \texttt{fname.out}in
the same directory. This is the only output file produced in the single-zone
escape probability case. When the CEP option is used, additional files are
generated as output, with the intermediate extension \texttt{CEP} added to the
name of the input file. For instance, for the
\texttt{Samples/CO/Flower\_rates.inp} input file, the following output files
are generated:
\begin{itemize}
\item
\texttt{Samples/CO/Flower\_rates.out} --- main output file
\item
\texttt{Samples/CO/Flower\_rates.CEP.trad} --- equivalent radiation
temperature for each transition and zone
\item
\texttt{Samples/CO/Flower\_rates.CEP.pop} --- population of each level for
each zone
\item
\texttt{Samples/CO/Flower\_rates.CEP.slab} --- final slab partitioning in
the converged solution
\item
\texttt{Samples/CO/Flower\_rates.CEP.flux} --- line profiles of emergent
flux in each selected transition
\end{itemize}

%%%%%%%%%%%%%%%%%%%%%%%%%%%%%%%%%%%%%%%%%%%%%%%%%%
%%%%%%%%%%%%%%%%%%%%%%%%%%%%%%%%%%%%%%%%%%%%%%%%%%
\section{Molecular Data and the BASECOL Database}
\label{sec:basecol}


As explained above, molecular data can be placed anywhere provided the root
directory where the file is located is indicated in each input file. The directory
\texttt{DataBase} contains a set of files with data for a few molecules. 

\subsection{Energy levels and radiative transitions}
The energy levels and radiative transitions of a model molecule are specified in the
corresponding file \texttt{mol\_name.molecule}.  For example, here we shown an extract
of \texttt{CO.molecule}:

\newpage

\begin{verbatim}
Rotational energy levels for the ground state of CO
Reference: CDMS

N. levels and molecular mass
>
41  28.d0

     N        g      Energy in cm^{-1}    Level details
>
     1        1           0.0000           'J =   0'
     2        3           3.8450           'J =   1'
     3        5          11.5350           'J =   2'
     4        7          23.0695           'J =   3'
     5        9          38.4481           'J =   4'
     6       11          57.6704           'J =   5'
...
    39       77        2835.7627           'J =  38'
    40       79        2984.2707           'J =  39'
    41       81        3136.5095           'J =  40'

Einstein coefficients A_ij for CO
Reference: CDMS

  i         j         A_ij in s^{-1}
>
  2         1         7.203e-08
  3         2         6.910e-07
  4         3         2.497e-06
  5         4         6.126e-06
  6         5         1.221e-05
  7         6         2.137e-05
  8         7         3.422e-05
...
 39        38         4.120e-03
 40        39         4.365e-03
 41        40         4.613e-03
\end{verbatim}
The listing must be ascending in energy.  The statistical weight $g$ is entered
as integer, energy is in cm$^{-1}$. Each level is internally identified in \M\
by the running number listed in the first column. The level details are all
read by \M\ as a single string and used only for information in the output.
All the lines before the line containing only the symbol ``\texttt{>}'' are
treated as header. The same applies to the rest of files defining the energy
levels, radiative and collisional transitions.
After the energy levels, we find a header for the radiative transitions
and the list of Einstein coefficients for spontaneous emission ($A_{ij}$).


\subsection{Collisional Rates}
As already mentioned (see \S\ref{sec:collisions}), there are several options
for specifying collision rates. The most common is to supply tabulations in a
file. The files with the collisional rates are typicall placed inside the directory 
\texttt{Coll/}. You can use a Python program to generate the file \texttt{list\_collisions.dat}
with all the collisional files in the directory:
\begin{verbatim}
python listCollisions.py
\end{verbatim}

As an example, here are the first few lines from \texttt{OH\_H2o.kij}:
\begin{verbatim}
Collision rate coefficients for OH-H2_ortho collisions.
Reference: Offer et al., 1994, J. Chem.Phys., 100, 362

>

Number of temperature columns = 5

   I   J                         TEMPERATURE (K)

              15.0        50.0        100.0       150.0       200.0

   2   1    0.40E-10    0.39E-10    0.34E-10    0.31E-10    0.28E-10
   3   1    0.17E-09    0.24E-09    0.26E-09    0.25E-09    0.24E-09
   3   2    0.58E-10    0.71E-10    0.72E-10    0.69E-10    0.65E-10
   4   1    0.35E-10    0.43E-10    0.43E-10    0.41E-10    0.39E-10
\end{verbatim}
The rate coefficients are entered in cm$^3$s$^{-1}$.  Only downward
(de-excitation) rates need be specified. The program accounts for excitation
rates via the Boltzmann detailed-balance relation. Elastic collisions are
ignored.

\subsection{The LAMBDA Database}
We provide a Python (Python 2.7) script to download data from the Leiden Atomic and Molecular Database (LAMDA)\footnote{Accessible at \texttt{http://home.strw.leidenuniv.nl/$\sim$moldata/}.}
and transform it to the MOLPOP-CEP format. This way, adding new molecules to perform calculations with 
our code turns out to be straightforward. The LAMDA database contains information for 3 atomic and an increasing number of molecular species. The
script is placed on the \texttt{DataBase} directory and can be invoked as:
\begin{verbatim}
python lamda.py
\end{verbatim}
It will list the molecules currently available in the database and ask for which ones to download.

\subsection{The Basecol Database}
The Basecol bibliographic and numerical database was established at Meudon
Observatory to address the community needs for data on molecular excitations.
Accessible at \url{http://basecol.obspm.fr/}, Basecol stores extensive
information on molecular frequencies, transition rates and collisional
excitations. The Basecol team has a web tool
that remotely accesses their database and creates atomic and molecular data
files in the \M\ input format on the user's local computer. Since the automatic
access to the database is not simple, we distributed the code with the
database downloaded in 2009.


%%%%%%%%%%%%%%%%%%%%%%%%%%%%%%%%%%%%%%%%%%%%%%%%%%
%%%%%%%%%%%%%%%%%%%%%%%%%%%%%%%%%%%%%%%%%%%%%%%%%%

\newpage
\appendix

\section{APPENDIX}

%%%%%%%%%%%%%%%%%%%%%%%%%%%%%%%%%%%%%%%%%%%%%%%%%%
\subsection{Formulation of the problem}

The standard multilevel radiative transfer problem in a given domain requires
the joint solution of the radiative transfer (RT) equation (which describes the
radiation field) and the kinetic equations (KE) for the atomic or molecular
level populations (which describe the excitation state). In the most general
case, the numerical solution of this non-local and non-linear problem requires
to discretize the model atmosphere in $NP$ zones where the physical properties
are assumed to be known. The standard multilevel RT problem consists in
obtaining the populations, $n_j$, of each of the $j=1,2,\ldots,NL$ levels
included in the atomic or molecular model that are consistent with the
radiation field created inside the complete domain. This radiation field
contains contributions from possible background sources and from the radiative
transitions in the given atomic/molecular model.

Making the usual assumption of statistical equilibrium, the rate equation for each
level $i$ at each spatial point reduces to \citep[e.g.,][]{socas_trujillo97}:
\begin{equation}
\label{eq:statis_equil_eq}
\sum_{j < i}{\Gamma}_{ji}\,-\,\sum_{j > i}{\Gamma}_{ij}\,
+\,\sum_{j \neq i}{n_j C_{ji}} - n_i \sum_{j \neq i}{C_{ij}} = 0,
\end{equation}
where $C_{ij}$ is the so-called collisional rate that quantifies the number of
transitions per unit volume and unit time between levels $i$ and $j$. The symbol
$\Gamma_{lu}$ stands for the net radiative rate, which quantifies the net number of
radiative transitions between a bound lower level $l$ to an upper level $u$. Its
expression in terms of the populations of the upper and lower levels is:
\begin{equation}
\label{eq:net_rad_rate}
\Gamma_{lu}\,=\,n_l R_{lu}\,-\,n_u\,R_{ul}\, ,
\end{equation}
where $R_{ij}$ are the radiative rates.

The collisional rates are assumed to be known and given by the local physical
conditions of the atmosphere. On the other
hand, the net radiative rates $\Gamma_{lu}$ depend on the radiation field that is
present in the domain. For bound-bound transitions, the following expression
holds:
\begin{equation}
\label{eq:net_rate_bound}
\Gamma_{lu}\,=(n_lB_{lu}-n_uB_{ul}){\bar{J}}_{lu}\,-\,n_uA_{ul},
\end{equation}
where $A_{ul}$, $B_{ul}$ and $B_{lu}$ are the spontaneous emission, stimulated emission and
absorption Einstein coefficients, respectively. The radiation field is parameterized
in terms of the mean intensity $\bar J_{lu}$, which
is the frequency averaged mean intensity weighted by the line absorption profile:
\begin{equation}
\label{eq:mean_intensity}
\bar J_{lu} = \frac{1}{4 \pi} \int d \mathbf{\Omega} \int d \nu \phi_{lu} (\nu,
\mathbf{\Omega})
I_{\nu \mathbf{\Omega}},
\end{equation}
where $\phi_{lu}(\nu,\mathbf{\Omega})$ and $I_{\nu \bf \Omega}$ are, respectively, the normalized line
profile and
the specific intensity at frequency $\nu$ and direction $\bf \Omega \rm$. The
specific intensity is governed by the radiative
transfer equation, which can be formally solved if we know the variation of the
opacity ($\chi_{\nu}$) and of the source function ($\epsilon_{\nu}/\chi_{\nu}$, being
$\epsilon_{\nu}$ the emission coefficient)
in the medium. Once the stellar atmosphere is discretized, the specific intensity
can be written formally as
\begin{equation}
\label{eq:formal_sol}
\bf I_{\nu \mathbf{\Omega}} = \bf \Lambda_{\nu \mathbf{\Omega}} \left[ \bf S_{\nu}
\right] + \mathcal{\bf T}_{\nu \mathbf{\Omega}}, \nonumber
\end{equation}
where $\mathcal{\bf T}_{\nu \mathbf{\Omega}}$ is a vector that accounts for the
contribution of the boundary conditions to the intensity
at each spatial point of the discretized medium, $\bf S_{\nu}$ is the source
function vector
and $\bf \Lambda_{\nu \mathbf{\Omega}}$ is an operator whose element
$\Lambda_{\nu\mathbf{\Omega}}(i,j)$ gives the response of the radiation field at
point ``$i$'' due to a unit-pulse perturbation in the source function at point ``$j$''.

Since the radiative transfer equation couples different parts of the atmosphere and
the absorption and emission properties at all the spatial points depend on the level
populations, the RT problem is both \emph{non-local} and \emph{non-linear}.
Therefore, the system of Eqs. (\ref{eq:statis_equil_eq}) represents a highly
non-linear problem. This non-linearity makes it necessary to apply suitable iterative
methods. Among them, the simplest one is the $\Lambda$-iteration \citep[e.g.,][]{mihalas78},
in which, starting from an initial estimation of the populations, the mean intensity at each
transition is obtained through Eq. (\ref{eq:mean_intensity}) and plugged into
Eqs. (\ref{eq:statis_equil_eq}). The resulting linear system is solved and the iteration is
repeated. This scheme works correctly when the optical depth of all transitions is
less than unity but the convergence time increases dramatically if any transition is optically
thick. The reason is that the iterative scheme transfers information in the domain in steps
of the order of $\tau \approx 1$. In optically thick cases, it takes many iterations to
transfer information from one part of the domain to the others. The accelerated $\Lambda$-iteration
\citep{rybicki_hummer91,rybicki_hummer92} overcomes many of the convergence problems
of the simple $\Lambda$-iteration by rewriting the net radiative rates of
Eq. (\ref{eq:net_rate_bound}) using a suitable combination of population from the
previous and the new iterative step, thus leading to an important improvement in
the convergence properties. A modification of this method by \cite{trujillo_fabiani95}
leads to the Gauss-Seidel and Successive Overrelaxation methods that produce an
improvement in the convergence rate of up to an order of magnitude. Finally, methods
based on multigrid iterations \cite{steiner91,fabiani_trujillo_auer97} have also
been applied to the radiative transfer problem with success.

%%%%%%%%%%%%%%%%%%%%%%%%%%%%%%%%%%%%%%%%%%%%%%%%%%
\subsubsection{Coupled Escape Probability} Our code solves the radiative
transfer problem under the approximation of a plane-parallel slab whose
physical properties vary only perpendicular to the surface.  The optical depth
along a ray slanted at $\theta = \cos^{-1} \mu$ from normal is $\tau_\nu(\mu) =
\tau\Phi(x)/\mu$, and the intensity along the ray obeys the radiative transfer
equation:
\begin{equation}
\label{eq:rad_tran}
     \mu{dI_\nu(\tau, \mu)\over d\tau} = \Phi(x)[S(\tau) - I_\nu(\tau, \mu)].
\end{equation}
It is useful to introduce a quantity called the net radiative
bracket \citep{athay_skumanich71}, defined by:
\begin{equation}
\label{eq:p1}
   p(\tau) = 1 - {\Jbar(\tau)\over S(\tau)},
\end{equation}
which plays the role of a escape probability so that the mean intensity at one point
is just $(1-p)$ times the local source function. In the standard one-zone
case, this is equivalent to the well-known escape probability. If the slab
is divided into many zones, this factor takes into account correctly the
no-local character of the radiative transfer. From the formal solution of
the radiative transfer equation,
\begin{equation}
\label{eq:p2}
   p(\tau) = 1 - {1\over 2S(\tau)}\int_0^{\tot}\!\!\! S(t)dt
                 \int_{-\infty}^{\infty}\!\!\!\Phi^2dx
                \int_0^1\!\!\! e^{-|\tau - t|\Phi/\mu}{d\mu\over\mu}
\end{equation}
when there is no external radiation entering the slab.

Instead of the usual iterative scheme that consider separately the
radiative transfer equation and the statistical equilibrium equations,
it is possible to substitute Eq. (\ref{eq:p2}) into the net
radiative rate, resulting in:
\begin{equation}
\label{eq:netrate_withp}
\Gamma_{lu}\,=- n_u A_{ul} p(\tau),
\end{equation}
which demonstrates that the only radiative quantity actually needed for the calculation of
level populations at every position is the net radiative bracket $p(\tau)$.
Once this factor is known at each zone of the domain, we could compute the
level populations that are consistent with the radiation they produce without solving
for the intensity. It is evident from Eq. \ref{eq:p2} that the factor $p(\tau)$ itself
can be computed from the level populations, again without solving  for the
intensity. Therefore, inserting $p(\tau)$ from equation \ref{eq:p2} into the
radiative rate terms produces {\em level population equations that
properly account for all the effects of radiative transfer without actually
calculating the intensity itself}.

\section{Numerical implementation}
A numerical solution of the resulting level population equations requires a
spatial grid, partitioning the source into zones such that all properties can
be considered uniform within each zone. The degree of actual deviations from
uniformity, and the accuracy of the solution, can be controlled by decreasing
each zone size through finer divisions with an increasing number of zones.
Assume that the slab is partitioned into $z$ zones. The $i$-th zone, $i = 1
\dots z$, occupies the range $\tau_{i-1} < \tau \le \tau_{i}$, with $\tau_0=0$
and $\tau_z=\tau_t$, with $\tau_t$ the total optical depth. The optical depth
between any pair of zone boundaries is
\begin{equation}
\label{eq:tij}
    \t(i,j) = |\tau_i - \tau_j|
\end{equation}
so that the optical thickness of the $i$-th zone is $\t(i,i - 1)$.
One has to remind that the optical thickness of the $i$-th zone in a
given bound-bound transition under the assumption of complete redistribution
is given by the following linear combination
of the populations of the upper and lower levels:
\begin{equation}
\tau_i(\nu) = \frac{h \nu}{4\pi}
\left( n_l B_{lu} - n_u B_{ul} \right)
\Phi \left( \frac{\nu-\nu_0}{\Delta \nu_D^i} \right)
\end{equation}

Since the temperature and the density of colliders (H$_2$, H or e$^-$) is
assumed to be constant within each zone, the collisional rates $C_{ij}$ are
constant in the zone. Correspondingly, the net radiative rate is just given by
\begin{equation}
\label{eq:rate}
\Gamma_{lu}^i\,= -n_u^i A_{ul} p^i,
\end{equation}
where the population of the upper level in each zone is also constant and
the superscript $i$ is used as a zone label. Since the factor $p(\tau)$ varies
in the zone, it has to be replaced by a constant $p^i$ that should adequately
represent its value there, for example $p^i = \half\left[p(\tau_{i}) +
p(\tau_{i-1})\right]$ or $p^i = p\left(\half[\tau_{i} +\tau_{i-1}])\right)$.
There are no set rules for this replacement other than it must obey $p^i \to
p(\tau_{i})$ when $\t(i,i-1) \to 0$. We choose for $p^i$ the zone average
\begin{equation}
\label{eq:avg}
      p^i = {1\over\t(i,i-1)}\int_{\tau_{i-1}}^{\tau_i} p(\tau)d\tau,
\end{equation}
which turns out to be one of the key ingredients of the success of
the coupled escape probability. The reason is that the value of $p^i$ used
in each zone is obtained as an average value inside the zone of the \emph{true}
variation of $p(\tau)$. The other possibilities assume a simple behavior (linear)
of the $p(\tau)$ function inside the zone.

\subsection{Internal radiation}
From Eq. (\ref{eq:p2}), calculation of $p^i$ requires an integration over the
entire slab, which can be broken into a sum of integrals over the zones. In
each term of the sum, the zone source function can be pulled out of the
$\tau$-integration so that
\begin{eqnarray}
    p^i &= &1 - {1\over2\t(i,i-1)S^i}
         \sum_{j = 1}^z S^j \times \non
         &&\int_{\tau_{i - 1}}^{\tau_{i}}\!\!\!d\tau
         \int_{\tau_{j - 1}}^{\tau_{j}}\!\!\! dt
                 \int_{-\infty}^{\infty}\!\!\!\Phi^2dx
                \int_0^1\!\!\! e^{-|\tau - t|\Phi/\mu}{d\mu\over\mu}
\end{eqnarray}
The remaining integrals can be expressed in terms of common functions. Consider
for example
\begin{eqnarray}
    \beta^i &=& 1 - {1\over2\t(i,i-1)}\times \non
         &&\int_{\tau_{i - 1}}^{\tau_{i}}\!\!\!d\tau
          \int_{\tau_{i - 1}}^{\tau_{i}}\!\!\!dt
                 \int_{-\infty}^{\infty}\!\!\!\Phi^2dx
                \int_0^1\!\!\! e^{-|\tau - t|\Phi/\mu}{d\mu\over\mu},
\end{eqnarray}
the contribution of zone $i$ itself to $p^i$. It is straightforward to show
that $\beta^i = \beta(\t(i,i - 1))$, where
\begin{equation}
\label{eq:beta}
  \beta(\tau) = {1\over\tau}\int_0^\tau \!\!\!dt
                            \int_{-\infty}^{\infty}\!\!\!\Phi(x)dx
                            \int_0^1 \!\!\!d\mu\, e^{-t\Phi(x)/\mu}
\end{equation}
This function was first introduced by \cite{capriotti65}. It represents the probability
for photon escape from a slab of thickness $\tau$, averaged over the photon
direction, frequency and position in the slab. The contribution of zone $j \ne
i$ to the remaining sum can be handled similarly, and the final expression for
the coefficient $p^i$ turns out to be very simple:
\begin{equation}
\label{eq:pi}
     p^i = \beta^i + {1\over\t(i,i-1)}\!\!
     \sum_{\stackrel{j = 1}{j \ne i}}^z \!\! {S^j\over S^i} M^{ij}
\end{equation}
where
\begin{equation}
\label{eq:M}
    M^{ij} = -\frac12(\a(i,j) - \a(i-1,j) - \a(i,j-1) + \a(i-1,j-1))
\end{equation}
and where $\a(i,j) = \t(i,j)\beta(\t(i,j))$. The quantity $\a(i,j)$ obeys
$\a(i,j) = \a(j,i)$ and $\a(i,i) = 0$, therefore $M^{ij} = M^{ji}$ and $M^{ii}
= \a(i,i-1)$. The
first term in the expression for $p^i$ is the average probability for photon
escape from zone $i$, reproducing one of the common variants of the escape
probability method in which the whole slab is treated as a single zone
\cite[e.g.,][]{krolik_mckee78}. The subsequent sum describes the effect on the level
populations in zone $i$ of radiation produced in all other zones. Each term in
the sum has a simple interpretation in terms of the probability that photons
generated elsewhere in the slab traverse every other zone and get absorbed in
zone $i$, where their effect on the level populations is similar to that of
radiation external to the slab.

Inserting the final expression for $p^i$ from Eq. (\ref{eq:pi}) into the rate terms
of Eq. (\ref{eq:netrate_withp}) in every zone produces a set of non-linear algebraic
equations for the unknown level populations $n_k^i$. The solution of these equations
yields the full solution of the line transfer problem by considering only level
populations. The computed populations are self-consistent with their internally
generated radiation even though the radiative transfer equation is not handled
at all.

\subsection{Numerical evaluation of the auxiliary functions}

Although the couple escape probability overcomes the solution of the radiative
transfer equation, it is necessary to evaluate the $\tau$-dependent auxiliary
function $\beta(\tau)$. In order to speed up the evaluation of these functions,
we tabulated it for 1000 points in $\tau$ and a spline interpolation routine is
used to calculate its values at any other value of $\tau$ not in the database.
We have verified that this approach gives computational times comparable to
those given by using the approximate formula of \cite{krolik_mckee78} but the
reached precision is much larger.


\subsection{External radiation}

The only effect of external radiation on the rate equations is to modify the
net radiative rate of the $i$-th zone as a consequence of the modification of the radiation field.
If $\bar J_\mathrm{int}^i$ is the mean intensity in the $i$-th zone produced by the slab
itself, the total radiation field is given by:
\begin{equation}
\bar J^i = \bar J_\mathrm{int}^i + \bar J_\mathrm{ext}^i
\end{equation}
where $\bar J_\mathrm{ext}^i$ is the zone average of the contribution of the
external radiation. When the external radiation corresponds to the emission
from dust which permeates the source, \Ji\ is simply the angle-averaged
intensity of the local dust emission in the $i$-th zone. When the external
radiation originates from outside the slab and has an isotropic distribution
with intensity \Ie\ ($= J_e$) in contact with the $\tau = 0$ face, then
\begin{equation}
  \Ji = \half J_e{1\over\t(i,i-1)} \left( \a(i,0) - \a(i-1,0) \right).
\end{equation}
If the radiation is in contact with the $\tau=\tau_t$, the expression turns out
to be:
\begin{equation}
  \Ji = \half J_e{1\over\t(i,i-1)} \left( \a(i,z) - \a(i-1,z) \right).
\end{equation}

\subsection{Solution of the final equations}

The solution method just described is exact in the sense that the discretized
equations are mathematically identical to the original ones when $\t(i,i-1) \to
0$ for every $i$. As is usually the case, the only approximation in actual
numerical calculations is the finite size of the discretization, i.e., the
finite number of zones. As a consequence, if one is interested in solving the
problem up to a given precision in the level populations, one should start with
an initial number of zones and stop when the relative change between one grid
and a refined one in which a suitable regridding strategy is applied is below a
predefined tolerance. MOLPOP-CEP allows the user to select whether to converge
the solution using grid refinement or just converge the problem in a predefined
grid.

The actual numerical solution of the set of Eqs. (\ref{eq:statis_equil_eq}) can
be carried out using two different techniques. The most straightforward is to use
a Newton method to solve the non-linear equations. The interesting key point
is that the Jacobian matrix can be calculated analytically because the dependence
of the radiation field on the population is known analytically. The second
possibility is to apply the $\Lambda$-iteration method. The diagonal of the
$\Lambda$ operator, that gives the response of the radiation field at zone
$j$ to a unitary perturbation of the source function at point $i$, can be
easily calculated under the previous formalism. This second method of solution
can be competitive when the numerical inversion of the Jacobian turns out
to be too time consuming.

In order to improve the convergence properties of the code, the following strategy
is applied the solution of the statistical equilibrium equations. This is of special
interest for the case in which the physical conditions are constant. Two possible
possibilities exist: start from a molecular column density in which all the lines
are optically thin ($\tau < 1$) or start from a molecular column density in which
all the lines are optically thick ($\tau \gg 1$). In the first case, the level
populations should be close to the optically thin solution and this can be chosen
as the initial condition. In the second case, the level populations should be close
to local thermodynamical equilibrium (LTE) and the Boltzmann distribution can be
assumed as an initial condition. Then, the molecular column density is either reduced
or increased in steps until some suitable limits in the column density are
reached. This strategy allows the code to increase the convergence behavior and, as
a subproduct, the solution is obtained for many intermediate problems.

If the final solution is far from any of the limiting cases (optically thin or
LTE populations), the Newton method can suffer from convergence problems. Another
remedy for this difficulty that we plan to investigate in the future is to carry out
several accelerated $\Lambda$-iterations starting from one of the limiting cases
(hopefully the closer to the final solution) and then apply the Newton method
for the solution refinement. Although the accelerated $\Lambda$-iteration can also
suffer from convergence problems if started far from the solution, they are less
important than for the Newton scheme.


\section{Cooling and emergent radiation}

Once the populations are found, radiative quantities can be calculated in a
straightforward manner from summations over the zones. The emerging intensity
at direction $\mu$ is
\begin{equation}
    I_\nu(\tot,\mu) = \sum_{i = 1}^z
     \left(e^{-\tau^{z,i}\Phi/\mu} -  e^{-\tau^{z,i - 1}\Phi/\mu}\right)S^i.
\end{equation}
The flux density emerging from each face of the slab obeys
\begin{eqnarray}
    F_\nu(\tot) &=& 2\pi\sum_{i = 1}^z
    \left[E_3(\tau^{z,i}\Phi^i) - E_3(\tau^{z,i - 1}\Phi^i)\right] S^i \non
    -F_\nu(0)  &=& 2\pi \sum_{i = 1}^z
    \left[E_3(\t(i - 1,0)\Phi^i) - E_3(\t(i,0)\Phi^i)\right] S^i
\end{eqnarray}
where $E_3$ is the third exponential integral \citep[e.g.,][]{abramowitz72}. The line profile
is given by:
\begin{equation}
\Phi^i = \Phi \left( \frac{\nu-\nu_0}{\Delta \nu_D^i} \right)
\end{equation}
The slab luminosity, given
by the expression:
\begin{equation}
\Lambda = 4 \pi \int_0^{\tot} \Delta \nu_D S(\tau)p(\tau)d\tau,
\end{equation}
is calculated after discretization with the following summation:
\begin{equation}
\Lambda = 4 \pi \sum_{i = 1}^z
 \left(\alpha^{i,0} - \alpha^{i-1,0} - \alpha^{z,i} + \alpha^{z,i-1}\right)S^i.
\end{equation}

\begin{thebibliography}{17}
\expandafter\ifx\csname natexlab\endcsname\relax\def\natexlab#1{#1}\fi

\bibitem[{{Abramowitz} \& {Stegun}(1972)}]{abramowitz72}
{Abramowitz}, M., \& {Stegun}, I.~A. 1972, {Handbook of Mathematical Functions}
  (New York: Dover)

\bibitem[{{Athay} \& {Skumanich}(1971)}]{athay_skumanich71}
{Athay}, R.~G., \& {Skumanich}, A. 1971, ApJ, 170, 605

\bibitem[{{Bieniek} \& {Green}(1983)}]{bieniek_green83}
{Bieniek}, R.~J., \& {Green}, S. 1983, ApJL, 265, L29

\bibitem[{{Capriotti}(1965)}]{capriotti65}
{Capriotti}, E.~R. 1965, ApJ, 142, 1101

\bibitem[{{Elitzur}(1992)}]{elitzur92}
{Elitzur}, M. 1992, {Astronomical Masers} (Dordrecht: Kluwer Academic
  Publishers)

\bibitem[{{Elitzur} \& {Asensio Ramos}(2006)}]{elitzur_asensio_cep06}
{Elitzur}, M., \& {Asensio Ramos}, A. 2006, MNRAS, 365, 779

\bibitem[{{Fabiani Bendicho} {et~al.}(1997){Fabiani Bendicho}, {Trujillo
  Bueno}, \& {Auer}}]{fabiani_trujillo_auer97}
{Fabiani Bendicho}, P., {Trujillo Bueno}, J., \& {Auer}, L.~H. 1997, A\&A, 324,
  161

\bibitem[{{Flower}(2001)}]{flower01}
{Flower}, D.~R. 2001, J. Phys. B, 34, 2731

\bibitem[{{Krolik} \& {McKee}(1978)}]{krolik_mckee78}
{Krolik}, J.~H., \& {McKee}, C.~F. 1978, ApJS, 37, 459

\bibitem[{{Lockett} \& {Elitzur}(1989)}]{lockett_elitzur89}
{Lockett}, P., \& {Elitzur}, M. 1989, ApJ, 344, 525

\bibitem[{{Lockett} \& {Elitzur}(1992)}]{lockett_elitzur92}
---. 1992, ApJ, 399, 704

\bibitem[{{Mihalas}(1978)}]{mihalas78}
{Mihalas}, D. 1978, in {Stellar Atmospheres}, Vol. 455

\bibitem[{{Rybicki} \& {Hummer}(1991)}]{rybicki_hummer91}
{Rybicki}, G.~B., \& {Hummer}, D.~G. 1991, A\&A, 249, 720

\bibitem[{{Rybicki} \& {Hummer}(1992)}]{rybicki_hummer92}
---. 1992, A\&A, 262, 209

\bibitem[{{Socas-Navarro} \& {Trujillo Bueno}(1997)}]{socas_trujillo97}
{Socas-Navarro}, H., \& {Trujillo Bueno}, J. 1997, ApJ, 490, 383

\bibitem[{{Steiner}(1991)}]{steiner91}
{Steiner}, O. 1991, A\&A, 242, 290

\bibitem[{{Trujillo Bueno} \& {Fabiani Bendicho}(1995)}]{trujillo_fabiani95}
{Trujillo Bueno}, J., \& {Fabiani Bendicho}, P. 1995, ApJ, 455, 646

\end{thebibliography}


% \bibliographystyle{apj}
% \bibliography{/scratch/Dropbox/biblio,apjmnemonic}


\end{document}
